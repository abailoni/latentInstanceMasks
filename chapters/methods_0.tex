% !TEX root = ../patchEmbeddings_review.tex

\section{The model}\label{sec:model}

In this section, we present the architecture of the proposed model. In Sec. \ref{sec:self_masks} and Sec. \ref{sec:encoding_masks} we define local self-probability masks and show how we compressed them to a lower latent space by using a Variational Auto-Encoder. Then, in Sec. \ref{sec:model_and_loss} we present the main convolutional model and the employed training loss. Finally, in Sec. \ref{sec:multiscale_patches} we show how we predicted self-probability masks at different scales.

\subsection{Local self-probability masks}\label{sec:self_masks}
This work proposes to distinguish between different object instances based on instance-aware pixel-pair affinities, which specifies whether two pixels belong to the same instance or not.
Given a pixel of the input image with coordinates $\mathbf{x}= (x_1, x_2)$, then a set of affinities to neighboring pixels within a $K\times K$ window is learned, where $K$ is an odd number. 
We define the $K\times K$-neighborhood of a pixel as:
\begin{equation}
\mathcal{N}_{K\times K} \equiv \mathcal{N}_{K} \times \mathcal{N}_{K}, \qquad \text{where} \quad \mathcal{N}_{K} \equiv \left\{-\frac{K-1}{2}, \ldots, \frac{K-1}{2}\right\}.
\end{equation}
 We then represent the affinities associated to pixel $\mathbf{x}$ as a local instance-aware self-probability mask, i.e. functions $\mathcal{M}_{\mathbf{x}}: \mathcal{N}_{K\times K} \rightarrow [0,1]$ describing the probability for each neighboring pixel to be in the same instance of pixel $\mathbf{x}$.

In total, if the input image has $H\times W$ pixels, we predict $K^2 \times H \times W$ affinities. We represent the training targets for these affinities as binary ground-truth masks $\hat{\mathcal{M}}_{\mathbf{x}}: \mathcal{N}_{K\times K} \rightarrow \{0,1\}$, which can be derived from a ground-truth instance label image $\hat{L}: H\times W \rightarrow \mathbb{N}$ according to:
\begin{equation}
\forall\, \mathbf{x}\in H\times W, \quad \forall\, \mathbf{n}\in \mathcal{N}_{K\times K} \qquad \hat{\mathcal{M}}_{\mathbf{x}}(\mathbf{n}) = 
\begin{cases}
1, \quad &\text{if } \hat{L}(\mathbf{x}) = \hat{L}(\mathbf{x}+\mathbf{n}) \\
0, \quad & \text{otherwise}. 
\end{cases}
\end{equation}
Note that these definitions can be easily generalized to 3D images.

\subsection{Encoding self-probability ground-truth masks}\label{sec:encoding_masks}
We first note that, among all feasible binary masks $\hat{\mathcal{M}}_{\mathbf{x}}: \mathcal{N}_{K\times K} \rightarrow \{0,1\}$, in practice only a part of them represent real occurring neighborhood structures. 
This suggests that it is possible to compress the masks to a lower dimensional space. 

In order to test this assumption, we then compress binary ground-truth masks $\hat{\mathcal{M}}_{\mathbf{x}}$ to latent variables $z_{\mathbf{x}}\in \mathbb{R}^Q$ by training a convolutional Variational Auto-encoder (VAE) \cite{kingma2013auto,rezende2014stochastic} consisting of an encoder $p_{\phi}(z_{\mathbf{x}}|\hat{\mathcal{M}}_{\mathbf{x}})$ and a decoder $p_{\phi}(\hat{\mathcal{M}}_{\mathbf{x}}|z_{\mathbf{x}})$.
In our experiments, we evaluate how the dimension $Q$ of the latent space impacts the quality of the reconstructed binary masks and find in this way an optimal latent space dimension that is compact enough but at the same time preserves most of the information contained in the binary masks.

\subsection{Predicting encoded self-probability masks} \label{sec:model_and_loss}
Given an input image, the model proposed in this work is trained to predict, for each pixel with coordinate $\mathbf{x}$, one latent vector $z_{\mathbf{x}}\in \mathbb{R}^Q$ encoding the associated self-probability mask $\mathcal{M}_{\mathbf{x}}$. Thus, if the input image has shape $H\times W$, the fully convolutional model outputs a feature map with shape $Q\times H\times W$. 
Note that this yields a much more compact model as compared to directly predicting all $K\times K$-masks with a final output feature map of size $K^2\times H\times W$.

Similarly to concurrent recent work \cite{hirsch2020patchperpix}, we test two ways of training the proposed model. \\

\textbf{Training VAE and backbone model separately} -- In this case, we first train a VAE to encode ground-truth binary masks as explained above in Sec. \ref{sec:encoding_masks}. 
The main backbone model is then trained to predict, for each pixel $\mathbf{x}$, the mean and the standard deviation of the encoded distribution $p_{\phi}(z_{\mathbf{x}}|\hat{\mathcal{M}}_{\mathbf{x}})$ predicted by the pre-trained encoder, where $\hat{\mathcal{M}}_{\mathbf{x}}$ is the ground truth self-probability mask associated to pixel $\mathbf{x}$. An L2 loss is used to pull the two encoded vectors close to each other's. 
The reasoning behind this approach is to train the backbone model to predict the masks in a meaningful compressed latent space. 
Nevertheless, as we will show in our experiments, this method was the least successful among the tested ones (\emph{similarly to the findings of \cite{hirsch2020patchperpix}}).\\
% L2 loss between the predicted latent vector of the model $\tilde{z}_i$ and the encoded version of the associated ground truth binary mask $p_{\phi}()$
% \begin{equation}
% \mathcal{J}^A_{\mathrm{Loss}} = \sum_{i\in H\times W} ()^2
% \end{equation}


\textbf{Training encoded self-probability masks end-to-end} --
In this second setup, the backbone model is trained end-to-end to predict a low-dimensional representation of the masks by applying a binary classification loss directly on decoded patches (see pipeline shown in Fig. \ref{fig:main_figure}). 
More specifically, we randomly sample $R$ pixels with coordinates $\mathbf{x}_1, \ldots, \mathbf{x}_R$ and their associated encoded masks $z_{\mathbf{x}_1}, \ldots, z_{\mathbf{x}_R}$ from the output of the backbone-model and retrieve the associated masks $\mathcal{M}_{\mathbf{x}_1}, \ldots, \mathcal{M}_{\mathbf{x}_R}$ by applying a convolutional decoder to each of them. 
The training loss is defined according to the S\o rensen-Dice coefficient \cite{dice1945measures,sorensen1948method} formulated for fuzzy set membership values:
\begin{equation}
\mathcal{J}_{\mathrm{loss}} = \frac{\sum_{i=1}^R \sum_{\mathbf{n} \in \mathcal{N}_{K\times K}} \big(1-\mathcal{M}_{\mathbf{x}_i}(\mathbf{n})\big)\cdot \big(1-\hat{\mathcal{M}}_{\mathbf{x}_i}(\mathbf{n})\big)}{\sum_{i=1}^R \sum_{\mathbf{n}\in \mathcal{N}_{K\times K}} \Big( \big(1-\mathcal{M}_{\mathbf{x}_i}(\mathbf{n})\big)^2 + \big(1-\hat{\mathcal{M}}_{\mathbf{x}_i}(\mathbf{n})\big)^2 \Big)}.
\end{equation} 
This type of loss, as compared to L2 or binary cross-entropy losses, is more robust against class imbalance and delivered better results for our applications.
Note that, in this setup, no convolutional encoder is trained.

\subsection{Multi-scale self-probability masks}\label{sec:multiscale_patches}
\TODO{To be completed}
As architecture for the main backbone model, we used a 3D U-Net \cite{ronneberger2015u, cciccek20163d}, which is one of the most common choices for applications to biological and medical images.
As we show in Fig. \ref{fig:main_figure}, the model is trained to predict latent vectors $z_{\mathbf{x}}$ at different depth levels of the UNet-decoder path...


