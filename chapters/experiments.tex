% !TEX root = ../patchEmbeddings_review.tex

\section{Experiments on neuron segmentation}
\subsection{Data: CREMI challenge} \label{sec:cremi_challenge}
We evaluate and compare our method on the task of neuron segmentation in electron microscopy (EM) image volumes. This application is of key interest in connectomics, a field of neuro-science with the goal of reconstructing neural wiring diagrams spanning complete central nervous systems. Currently, only proof-reading or manual tracing yields sufficient accuracy for correct circuit reconstruction \cite{schlegel2017learning}, thus further progress is required in automated reconstruction methods.

We test our method on the competitive CREMI 2016 EM Segmentation Challenge \cite{cremiChallenge} that is currently the neuron segmentation challenge with the largest amount of training data available. The dataset comes from serial section EM of \emph{Drosophila} fruit-fly tissue and consists of 6 volumes of $1250\times 1250\times 125$ voxels at resolution $4\times 4\times 40$ nm$^3$, three of which come with publicly available training ground truth. 
We achieved the best scores by downscaling the resolution of the EM data by a factor $(\frac{1}{2},\frac{1}{2},1)$, since this helped increasing the 3D context provided as input to the model.
We use the second half of CREMI sample C as validation set for our comparison experiments in Table \ref{tab:val_results} and then we train a final model on all the three samples with available ground truth labels to submit results to the leader-board in Tab. \ref{tab:test_results}. 
Results  are evaluated using the CREMI score, which is given by the geometric mean of Variation of Information Score (VOI split + VOI merge) and Adapted Rand-Score (Rand-Score), two popular metrics used to evaluate clusterings \cite{arganda2015crowdsourcing}.

\textbf{Data augmentation} -- The data from the CREMI challenge is highly \linebreak anisotropic and contains artifacts like missing sections, staining precipitations and support film folds. 
To alleviate difficulties stemming from misalignment, we use a version of the data that was elastically realigned by the challenge organizers with the method of \cite{saalfeld2012elastic}.
In addition to the standard data augmentation techniques of random rotations, random flips and  elastic deformations, we simulate data artifacts.
In more detail, we randomly zero-out slices, introduce alignment jitter and paste artifacts extracted from the training data. Both \cite{funke2018large} and \cite{lee2017superhuman} have shown
that these kinds of augmentations can help to alleviate issues caused by EM-imaging artifacts. For zero-out slices, the model is trained to predict the ground-truth labels of the previous slice.
On the test samples, we run predictions for overlapping volumes and then average them.


 
 


\subsection{Architecture details of the tested models}
As a backbone model we use a 3D U-Net consisting of a hierarchy of four feature maps with anisotropic downscaling factors $(\frac{1}{2},\frac{1}{2},1)$, similarly to \cite{lee2019learning,lee2017superhuman,wolf2018mutex}. 
% U-Net with three levels in the feauter map hierarchy, involving downscaling factors $(1, \frac{1}{2},\frac{1}{2})$
Models are trained with Adam optimizer and batch size equal to one. Before applying the loss, we slightly crop the predictions to prevent training on borders where not enough surrounding context is provided. 
See \TODO{supplementary material} for all details about the used architecture. 

 
\textbf{Baseline model (SNB)} -- As a strong baseline, we re-implement the current state of the art and train a model to predict affinities for a sparse neighborhood structure (Fig. \ref{fig:main_figure}a). We perform deep supervision by attaching three \emph{\sparseBr branches} (SNB) at different levels in the hierarchy of the UNet decoder and train the coarser feature maps to predict longer range affinities. Details about the used neighborhood structures can be found in \TODO{supplementary material}.

\textbf{Proposed model (ENB)} -- We then train a model to predict encoded \maskname masks (Fig. \ref{fig:main_figure}c). Similarly to the baseline model, we provide deep supervision by attaching four \emph{\encBr branches} (ENB) to the backbone U-Net. As explained in Sec. \ref{sec:multiscale_patches}, all branches predict 3D masks of shape $7 \times 7 \times 5$, but at different resolutions $(1,1,1)$, $(\frac{1}{4},\frac{1}{4},1)$ and $(\frac{1}{8},\frac{1}{8},1)$, as we show in \TODO{supplementary material}.

\textbf{Combined model (SNB+ENB)} -- Finally, we also train a combined model to predict both \maskname masks and a sparse neighborhood of affinities, by providing deep supervision both via \emph{\encBr} and \emph{\sparseBr} \emph{branches}. The backbone of this model is then trained with a total of seven branches: three branches equivalent to the ones used in the baseline model SNB, plus four additional ones like those in the ENB model.  

\subsection{Graph partitioning methods} 
Given the output of the compared models, we compute affinities $a_e$ either with the efficient method of Sec. \ref{sec:efficient_affs} (\textbf{EffAff}) or the average aggregation introduced in Sec. \ref{sec:aggr_affs} (\textbf{AggrAff}). 
The result of either is a signed pixel grid-graph, i.e. a graph with positive and negative edge weights that needs to be partitioned into instances. 
The used neighborhood connectivity of the graph is given in \TODO{supplementary material}. Positive and negative edge weights $w_e$ are computed by applying the additive transformation $w_e=a_e-0.5$ to the predicted affinities.


To obtain final instances, we test different partitioning algorithms.
 % that have been already applied to neuron segmentation. 
The Mutex Watershed (\textbf{MWS}) \cite{wolf2018mutex} is an efficient algorithm to partition graphs with both attractive and repulsive weights without the need for extra parameters. It can easily handle the large graphs considered here with up to $10^8$ nodes/voxels and $10^9$ edges\footnote{Among all edges given by the chosen neighborhood structure, we add only 10\% of the long-range ones, since the Mutex Watershed was shown to perform optimally in this setup \cite{bailoni2019generalized,wolf2018mutex}.}. 

Then, we also test another graph partitioning pipeline that has been often applied to neuron segmentation because of its robustness. This method first generates a 2D super-pixel over-segmentation from the model predictions and then partitions the associated region-adjacency graph to obtain final instances. Super-pixels are computed with the following procedure: first, the predicted direct-neighbor affinities are averaged over the two isotropic directions to obtain a 2D neuron-membrane probability map; then, for each single 2D image in the stack, super-pixels are generated by running a watershed algorithm seeded at the maxima of the boundary-map distance transform (\textbf{WSDT}). Given this initial over-segmentation, a 3D region-adjacency graph is built, so that each super-pixel is represented by a node in the graph. Edge weights of this graph are computed by averaging short- and long-range affinities over the boundaries of neighboring super-pixels. 
Finally, the graph is partitioned by applying the average agglomeration algorithm proposed in \cite{bailoni2019generalized} (\textbf{GaspAvg}).

% GaspAverage is more robust to noise as compared to Mutex Watershed \cite{bailoni2019generalized}, but it is considerably more computationally expensive on large graphs like the ones considered here (up to $10^8$ nodes/voxels and $10^9$ edges). Thus, in our comparison on the validation set, we also... 

  





% \begin{figure}[t]
%         \centering
% \begin{minipage}[t]{0.49\textwidth}
%     \centering
%     % \scriptsize
%         \begin{tabular}[t]{l|c}
%          Method & \makecell{CREMI-Score \\(lower is better)} \\ \midrule 
% \textbf{} \textbf{Average}& \textbf{0.226}  \\
%  Sum + Constraints \cite{levinkov2017comparative} & 0.282 \\
%  Abs. Max. \cite{wolf2018mutex} & 0.322 \\
%  Max. + Constraints & 0.324 \\
%  Sum \cite{keuper2015efficient} & 0.334 \\
%  Average + Constraints & 0.563 \\
% THRESH & 1.521 \\ 
%         \end{tabular}
%     % \captionof{table}{CREMI-Scores achieved by different linkage criteria and thresholding. All methods use the affinity predictions from our CNN as input. Scores are averaged over the three CREMI training datasets.}
%     \label{tab:results_cremi_train}
% \end{minipage}\hfill
% % \begin{minipage}[t]{0.35\textwidth}
% % \centering
% %     \scriptsize
% %     \vspace*{-1.5em}
% % \begin{tabular}[t]{l|cc}
% %         \multirow{2}{*}{Method}    & AP  & AP 50\% \\ 
% %          & \multicolumn{2}{c}{(higher is better)} \\ \midrule
% %            Panoptic-DeepLab \cite{cheng2019panopticdeeplab} & 34.6 & 57.3 \\
% %            UPSNet \cite{xiong2019upsnet} $\dagger$ & 33.0 & 59.6 \\
% %            SSAP \cite{Gao_2019_ICCV} & 32.7 & 51.8 \\
% %            AdaptIS \cite{sofiiuk2019adaptis} & 32.5 & 52.5 \\
% %            PANet \cite{liu2018path} $\dagger$ & 31.8 & 57.1 \\
% %            \textbf{GMIS Model \cite{liu2018affinity} +  Average} & \textbf{28.3} & \textbf{47.0} \\ 
% %            JOSECB \cite{neven2019instance} & 27.7 & 50.9 \\
% %            \textbf{GMIS} \cite{liu2018affinity} & \textbf{27.3} & \textbf{45.6} \\
% %            Mask R-CNN \cite{he2017mask} $\dagger$ & 26.2 & 49.9 \\
% %            SGN \cite{liu2017sgn} & 25.0 & 44.9 \\
% %            % DIN \cite{arnab2017pixelwise} & 20.0 & 38.8 \\
% %            % DWT \cite{bai2017deep} & 19.4 & 35.3 \\
% %            % InstanceCut \cite{kirillov2017instancecut} & 13.0 & 27.9 \\
% %         \end{tabular}
% %     % \caption{CityScapes \emph{test} set}
% %     % \vspace*{0.6em}
% %     \captionof{table}{Results on CityScapes test. Methods marked with~$\dagger$ are \emph{proposal-based}. Only methods that do not use external training data (such as MS COCO) are shown.}\label{tab:results_cityscapes}
% %     \label{tab:results_cityscapes_test}
% % \end{minipage}
% \end{figure}

\subsection{Results and discussion}

\textbf{Pre-training of the encoded space} -- The proposed model based on a \emph{\encBr branch} can be properly trained only if the dimension $Q$ of the latent space is large enough to encode all possible occurring neighborhood patterns. 
To make sure of this, we then trained a convolutional Variational Auto-encoder (VAE) \cite{kingma2013auto,rezende2014stochastic} to compress binary ground-truth \maskname masks $\hat{\mathcal{M}}_{\coord{u}}$ into latent variables $z_{\coord{u}}\in \mathbb{R}^Q$ and evaluated the quality of the reconstructed binary masks via the reconstruction loss. We then concluded that $Q=32$ was large enough to compress the masks considered here consisting of $7\times 7 \times 5=245$ pixels. 

As a first experiment, we tried to make use of this pre-trained encoded space to train the proposed \emph{\encBr branch} and predict encoded masks directly in this space by using a L2 loss on the encoded vectors. However, similarly to the findings of \cite{hirsch2020patchperpix}, this approach performed worse than directly training the full model end-to-end as described in Sec. \ref{sec:encoding_masks}. \\
% \begin{itemize}
% \item Mention how we tested the latent space dimension by training a VAE (or AE) to compress binary ground-truth \maskname masks: \emph{We then test this assumption by compressing binary ground-truth \maskname masks $\hat{\mathcal{M}}_{\coord{u}}$ to latent variables $z_{\coord{u}}\in \mathbb{R}^Q$ by training a convolutional Variational Auto-encoder (VAE) \cite{kingma2013auto,rezende2014stochastic} consisting of an encoder $p_{\phi}(z_{\coord{u}}|\hat{\mathcal{M}}_{\coord{u}})$ and a decoder $p_{\phi}(\hat{\mathcal{M}}_{\coord{u}}|z_{\coord{u}})$.
% In our experiments, we evaluate how the dimension $Q$ of the latent space impacts the quality of the reconstructed binary masks and find in this way an optimal latent space dimension that is compact enough but at the same time preserves most of the information contained in the binary masks.}
% \item Mention that we also tried to pre-train the encoded space with a VAE, but this did not work better than directly training the space end-to-end (similarly to PatchPerPix). 
% \emph{In this case, we first train a VAE to encode ground-truth binary masks as explained above in Sec. \ref{sec:encoding_masks}. 
% The main backbone model is then trained to predict, for each pixel $\coord{u}$, the mean and the standard deviation of the encoded distribution $p_{\phi}(z_{\coord{u}}|\hat{\mathcal{M}}_{\coord{u}})$ predicted by the pre-trained encoder, where $\hat{\mathcal{M}}_{\coord{u}}$ is the ground truth \maskname mask associated to pixel $\coord{u}$. An L2 loss is used to pull the two encoded vectors close to each other's. 
% The reasoning behind this approach is to train the backbone model to predict the masks in a meaningful compressed latent space. 
% Nevertheless, as we will show in our experiments, this method was the least successful among the tested ones (\emph{similarly to the findings of \cite{hirsch2020patchperpix}}).}
% \end{itemize}



\textbf{Training encoded masks} -- As we show in our validation experiments in Table \ref{tab:val_results}, all models trained to predict \maskname masks achieved better scores than the strong baseline SNB representing the current state of the art method predicting affinities for a sparse neighborhood structure. 
% Predicting only affinities for a sparse-neighborhood, like the SBN model, is an easier task as compared to predicting large and dense neighborhood structures. 
Our interpretation of this result is that using the encoding process to predict \maskname masks encourages the model to predict more consistent neighborhood patters by learning prior knowledge about the shape of the segments, which can be helpful to correctly segment the most difficult regions of the data. \\
% Thus, we argue that the combined model is less likely to result in a basic boundary detector based on simple local features and it is instead more likely to learn more complex features and use for example prior information about the shape of the segments.
 % like in the models ENB and ENB+SBN, is a more challenging task than predicting affinities for a sparse-neighborhood, like for the SBN model. Thus, we argue that the models predicting \maskname masks are 
 % and instead are encouraged to learn long-range features that can be helpful in the most challenging cases.
% Moreover, ... 
% Not only improves affinities, but also the opposite. 

\begin{table}[t]
\small
\centering
  % {\renewcommand{\arraystretch}{1.3}
  % \resizebox{\textwidth}{!}{
        \begin{tabular}[t]{l M{11.5em} c c }
         % Method & \makecell{CREMI-Score \\(lower is better)} \\ \midrule 
\thead[l]{Model} &  \thead{Partitioning \\algorithm}  & \thead{CREMI-Score \\(lower is better)} & \thead{VI-merge \\(lower is better)} \\ \toprule 

\textbf{ENB+SNB+AggrAff} & \textbf{MWS} & \textbf{0.153} & \textbf{0.272} \\
ENB+AggrAff & MWS & 0.184 & 0.273 \\
ENB+EffAff & MWS & 0.419 & 0.302 \\
ENB+SNB+EffAff & MWS & 0.532 & 0.447 \\
SNB+EffAff & MWS & 1.155 & 0.874 \\ \midrule
ENB+EffAff & WSDT+GaspAvg & \textbf{0.173} & \textbf{0.234} \\
ENB+SNB+EffAff & WSDT+GaspAvg & 0.237 & 0.331 \\
SNB+EffAff & WSDT+GaspAvg & 0.254 & 0.355 \\
ENB+SNB+AggrAff & WSDT+GaspAvg & 0.334 & 0.388 \\
ENB+AggrAff & WSDT+GaspAvg & 0.357 & 0.391 \\

% ENB+SNB & MutexWatershed & _affs &  &  &  & 0.585 & 0.257 & 0.454 & 0.878 \\
% SNB & MutexWatershed & _affs &  &  &  & 0.910 & 0.344 & 0.671 & 1.736 \\
% SNB & MutexWatershed & _affs_withLR &  &  &  & 1.126 & 0.487 & 0.766 & 1.837 \\
% ENB & WSDT+MC & 0.178 & 0.040 & 0.233 & 0.557 \\
% ENB+SNB & WSDT+MC & 0.241 & 0.082 & 0.331 & 0.377 \\
% SNB & WSDT+MC & 0.257 & 0.087 & 0.355 & 0.399 \\
% ENB+SNB+AggrAff & MutexWatershed & _affs &  &  &  & 0.269 & 0.086 & 0.370 & 0.470 \\
% ENB+AggrAff & MutexWatershed & _affs &  &  &  & 0.302 & 0.095 & 0.365 & 0.595 \\
% ENB+SNB & MutexWatershed & _affs_withLR &  &  &  & 0.416 & 0.145 & 0.385 & 0.803 \\


% % % & & SNB & 1.144 & 0.875 \\ 
% % & & SNB+ENB & \textbf{0.115} & 0.222 \\
% % & & SNB & 0.119 & \textbf{0.219} \\
% % GaspAverage  & & SNB+ENB+AggrAff &  0.156 & 0.277 \\
% % & &ENB & 0.173 & 0.242 \\
% % & &ENB+AggrAff & 0.188 & 0.281 \\ \midrule
% & &SNB+ENB & \textbf{0.130} & \textbf{0.234} \\
% WSDT+GaspAverage & & SNB & 0.147 & 0.258 \\
% && ENB & 0.173 & 0.240 \\ \midrule
% & &SNB+ENB & \textbf{0.135} & \textbf{0.235} \\
% WSDT+MC & & SNB &  0.151 & 0.258 \\
% & & ENB & 0.178 & 0.240 \\ \midrule
% & & SNB+ENB+AggrAff & \textbf{0.155} & 0.278 \\
% & & ENB+AggrAff & 0.183 & \textbf{0.275} \\
% MutexWatershed & & ENB & 0.417 & 0.311 \\
% & & SNB+ENB &  0.531 & 0.449 \\
% % & & SNB+ENB & 0.539 & 0.376 \\
% & & SNB & 0.895 & 0.634 \\ 
% % Superpixels without long-range:
% % UNet+SNB+ENB & WSDT+GASP-Avg & 0.137 & 0.260 \\
% % UNet+SNB & WSDT+GASP-Avg & 0.188 & 0.333 \\
% % UNet+ENB & WSDT+GASP-Avg & 0.197 & 0.299 \\
% % Multicut without long range:
% % UNet+SNB+ENB & WSDT+MC & 0.127 & 0.245 \\
% % UNet+ENB & WSDT+MC & 0.156 & 0.248 \\
% % UNet+SNB & WSDT+MC & 0.183 & 0.317 \\
% % Superpixels ong AggrAff
% % UNet+SNB+ENB+AggrAff & WSDT+GASP-Avg-LR & 0.219 & 0.309 \\
% % UNet+SNB+ENB+AggrAff & WSDT+GASP-Avg & 0.229 & 0.316 \\
% % UNet+ENB+AggrAff & WSDT+GASP-Avg-LR & 0.243 & 0.316 \\
% % UNet+ENB+AggrAff & WSDT+GASP-Avg & 0.248 & 0.323 \\


        \end{tabular}
        % }
        \vspace{1em}
        \caption{Comparison experiments on our CREMI validation set} \label{tab:val_results}
\end{table}


\textbf{Affinities from aggregated masks}  -- 
In our validation experiments of Table \ref{tab:val_results}, we also test the affinities computed by averaging over overlapping masks, as described in Sec. \ref{sec:aggr_affs}. We then partition the resulting signed graph by using the Mutex Watershed, which is a very simple parameter-free algorithm with empirical linearithmic complexity in the number of edges. 
Our experiments show that, for the first time on this type of more challenging neuron segmentation data, the Mutex Watershed (MWS) achieves better scores than the super-pixels-based methods (WSDT+GaspAvg), which have always been known to be more robust to noise but also require the user to tune more hyper-parameters.   

We also note that the MWS achieves competitive scores only with affinities computed from aggregated masks (AggrAff).
This shows that the MWS algorithm can take full advantage of the \maskname aggregation process by assigning the highest priority to the edges with most attractive and repulsive weights that were consistently predicted across overlapping masks.

On the other hand, most of the affinities directly predicted by the \emph{\sparseBr branch} (EffAff) are almost binary, i.e. they present values either really close to zero or really close one\footnote{In practice, this effect is even stronger when the binary classification loss is based on the S\o rensen-Dice coefficient like the one used here.}.
This is not an ideal setup for the MWS, which is a very greedy algorithm merging and constraining clusters according to the most attractive and repulsive weights in the graph.
In fact, in this setting the MWS can often lead to over-segmentation and under-segmentation artifacts like those observed in the output segmentations of the (ENB+SNB+EffAff) and (SNB+EffAff) models. Common causes of these mistakes can be for example inconsistent predictions from the model and partially missing boundary evidence, which are very common in this type of challenging application. 
% On the other hand, 
% Thus, we first note that these affinities are substantially different from those computed by averaging over overlapping masks as described in Sec. \ref{sec:aggr_affs} \TODO{Fig?}.
 % This fact is also shown in Table \ref{tab:val_results} by the validation scores achieved by the Mutex Watershed.

Finally, we also note that superpixel-based methods did not perform equally well on affinities computed from aggregated masks and the reason is that these methods were particularly tailored to perform well with the more \emph{binary-like} classification output of the \emph{\sparseBr branch} (EffAff). \\


\begin{table}[t]
\centering
\begin{minipage}[t]{\textwidth}
    \centering
    % \scriptsize
        \begin{tabular}[t]{L{18em} c M{9.8em}}
        Model & \thead{Partitioning\\algorithm} & \makecell{CREMI-Score \\(lower is better)}  \\ \midrule
\UPDATE{GaspUNet}+SNB+EffAff \cite{bailoni2019generalized} & WSDT+LMC  &  0.221\\
PNIUNet+SNB+EffAff \cite{lee2017superhuman} & \UPDATE{WSDT+GaspAvg}  & 0.228 \\
\UPDATE{GaspUNet}+SNB+EffAff \cite{bailoni2019generalized} & GaspAvg $\dagger$ & 0.241 \\
\textbf{OurUNet+ENB+SNB+AggrAff} &\textbf{MWS} $\dagger$ & \textbf{0.246} \\
\textbf{OurUNet+ENB+EffAff}& \textbf{WSDT+GaspAvg}  & 0.268 \\
MALAUNet+EffAff \cite{funke2018large} & WSDT+MC  & 0.276 \\
\textbf{OurUNet+ENB+SNB+EffAff} & \textbf{WSDT+GaspAvg} & 0.280 \\
CRUNet+EffAff \cite{zeng2017deepem3d} & \UPDATE{?} & 0.566  \\
LFC+EffAff \cite{parag2017anisotropic} & \UPDATE{?} & 0.616  \\
        \end{tabular}
        \vspace*{0.99em}
    \caption{Current leading entries in the CREMI challenge leaderboard \cite{cremiChallenge} (March 2020). Partitioning algorithms that do not rely on super-pixels are marked with $\dagger$.}
    \label{tab:test_results}
\end{minipage}
\end{table}

\textbf{Training both masks and affinities} -- In our experiments, the combined model, which was trained to predict both affinities and \maskname masks, achieved the best scores and yielded sharper and more accurate mask predictions (see scores from the ENB+SNB+AggrAff and ENB+AggrAff models).
In general, providing simultaneous loss for multiple tasks has often been proven beneficial in a supervised learning setting.
In our case, the better performances of the combined model could also be explained by the fact that, due to GPU-memory restrictions, the \emph{\encBr branch} was trained only with a sparse gradient from a small fraction of all the predicted encoded masks (see Sec. \ref{sec:encoding_masks}), whereas the \emph{\sparseBr branch} received gradient for every pixel in the output feature map. \\
% On the other hand, the combined model ENB+SNB, which receives dense gradient from the \emph{\sparseBr branch}, predicted sharper and more accurate \maskname mask as compared to the ENB model. Thus, to summarize, our results show that the combined model trained to predict both affinities and \maskname masks performs better than the models trained to solve the two tasks singularly, which is a common behavior often observed in supervised learning. \\





\textbf{Results on test samples} -- The evaluation on the three test samples presented in Table \ref{tab:test_results} confirms our findings from the validation experiments: the best scores are achieved by the combined model (ENB+SNB) and by using the Mutex Watershed algorithm (MWS) on affinities averaged over overlapping masks (AggrAff).
Our method achieves comparable scores to the only other method in the leader-board that does not rely on super-pixels and uses instead the average agglomeration algorithm GaspAvg proposed in \cite{bailoni2019generalized}. This algorithm has been shown to be more robust to noise as compared to Mutex Watershed, however it is also considerably more computationally expensive to run on large graphs like the ones considered here. 

The other leading entries in the leader-board are all given by methods that are very similar to our baseline model SNB, which in our validation experiments consistently performed worse.
A possible explanation of this result could be given by noting that all our trained models seem to particularly suffer from the \emph{glia-confusion problem}, that has been already observed in related work on neuron segmentation \TODO{\cite{lee2019learning,januszewski2018high}}. Glia (putative astrocyte) are segments that look quite different from other neurons and are usually considered separately in the final reconstructed diagram of neural circuit connectivity. They often present a main body with a lot of very thin processes that can be \emph{self-touching}. 
% glia segments are substantially different from the other neurons and are usually considered separately in the final reconstructed diagram of neural circuit connectivity. Glia segments usually present a main body with a lot of very thin glial processes
Locally, at these self-touching points, glial thin fragments look like distinct segments with clear membrane evidence separating them, even though in the full 3D volume they are connected and share then the same ground truth label.
When this happens during training, the model is initially confused and then usually tries to ``memorize'' these self-touching examples by predicting high affinity values despite the clear membrane evidence.
% When these self-touching examples happen on the training set, despite the clear neuron membrane evidence, the model is trained to predict the same 
% Every time there is a glia fragment in the training dataset presents two narrow processes that are self-touching, the model is trained to predict one single label for them, even though these processes locally appear like two distinct neurons with clear membrane evidence separating them.
In practice, this can often lead to wrong merge decisions on the validation and test set, because the model confuses distinct narrow neurons for self-touching fragments of the same glia and wrongly predict an high merge-affinity connecting them.
In future work, we plan to process this type of glial fragments differently, similarly to related work \TODO{\cite{lee2019learning,januszewski2018high}}.

 % The embedding net is confused by complex glial self-contacts in the training set (AC3). Top: the embedding net “memorizes” the self-touching glia in the training set (orange object, last column) by assigning uniform embeddings across the self-contacts (yellow arrowheads, second column), effectively erasing boundaries in the nearest neighbor metric graph (third column). Bottom: the embedding net makes a mistake on a similar-looking location in the training set where the contacts are between two distinct glia this time (yellow arrowheads, last column). As can be seen here, glia with complex morphology (putative astrocytes) make numerous self-contacts that are not properly separated by background voxels in the ground truth annotation. This becomes a significant source of noise during training, systematically compromising the embedding net’s performance around glia, even in the training set. For metric graph (third column), we show min(a x , a y ) where a x /a y are x/y nearest neighbor metric-derived affinities. To visualize embeddings, we used PCA to project the 24-dimensional embedding space onto the three dimensional RGB color space.

 % Shown here is a complex glia (putative astrocyte) from the extra test volume E1. The baseline segmentation (top) has systematic split errors concentrated on the very thin glial processes (yellow boxes), whereas object-centered representation of the dense embeddings successfully extends them. However, the proposed method makes a couple of split errors (green and cyan segments) due to the conservative repulsive constraints put by the Mutex Watershed on the self-touching glial fragments. Here mean embedding agglomeration was unable to perfectly heal the self-touching split errors, due to the failures of the heuristic for detecting self-touching split candidates.




